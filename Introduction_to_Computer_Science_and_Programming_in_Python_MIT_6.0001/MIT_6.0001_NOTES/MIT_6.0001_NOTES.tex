			%++++++++++++++++++++++++++++++++++++++++
% Don't modify this section unless you know what you're doing!
\documentclass[letterpaper,12pt]{article}
\usepackage{tabularx} % extra features for tabular environment
\usepackage{amsmath}  % improve math presentation
\usepackage{graphicx} % takes care of graphic including machinery
\usepackage[margin=1in,letterpaper]{geometry} % decreases margins
\usepackage{cite} % takes care of citations
\usepackage[final]{hyperref} % adds hyper links inside the generated pdf file
\hypersetup{
	colorlinks=true,       % false: boxed links; true: colored links
	linkcolor=blue,        % color of internal links
	citecolor=blue,        % color of links to bibliography
	filecolor=magenta,     % color of file links
	urlcolor=blue         
}
%++++++++++++++++++++++++++++++++++++++++


\begin{document}

\title{Introduction to Computer Science and Programming in Python (MIT)}
\author{N. Ceylan}
\date{\today}
\maketitle



\section{Lecture 1: What is Computation?}

% C:\Users\eiree\AppData\Local\Programs\Python\Python37-32\python.exe .\2.1.py



\begin{description}

	\item[$\bullet$ int] İntegers
	\item[$\bullet$ float] Real numbers
		\item[$\bullet$ bool] Boolean
	\item[$\bullet$ NoneType] Special none value
	\item[$\bullet$ type()] can use type() to see type of an object.
	
	
\end{description}

\section{ Branching and Iteration}




\section{String Manipulation, Guess and Check, Approximations, Bisection}


\begin{description}
	
	\item[$\bullet$ $>>>$] s = "abc"
	\item[$\bullet$ $>>>$] len(s)
    \item[$\bullet$ $>>>$] 3


	
	
\end{description}


\end{document}
