%++++++++++++++++++++++++++++++++++++++++
% Don't modify this section unless you know what you're doing!
\documentclass[letterpaper,12pt]{article}
\usepackage{tabularx} % extra features for tabular environment
\usepackage{amsmath}  % improve math presentation
\usepackage{graphicx} % takes care of graphic including machinery
\usepackage[margin=1in,letterpaper]{geometry} % decreases margins
\usepackage{cite} % takes care of citations
\usepackage[final]{hyperref} % adds hyper links inside the generated pdf file
\hypersetup{
	colorlinks=true,       % false: boxed links; true: colored links
	linkcolor=blue,        % color of internal links
	citecolor=blue,        % color of links to bibliography
	filecolor=magenta,     % color of file links
	urlcolor=blue         
}
%++++++++++++++++++++++++++++++++++++++++


\begin{document}

\title{Python for Data Science}
\author{N. Ceylan}
\date{\today}
\maketitle


\section{Introduction and Course Information}

\section{Week 1: Getting Started with Data Science}



Course   Week 1: Getting Started with Data Science   Engagement: How Data Science Happens   Video: How Does Data Science Happen


\section{Week 2: (Optional) Background in Python and Unix}

\section{Week 3 - Jupyter Notebooks and Numpy}



\section{Week 4 - Pandas}


\section{Week 5 - Data Visualization
}


\section{Week 6 - Mini Project Week
}

\section{Week 7 - Introduction to Machine Learning
}
\section{Week 8 - Working with Text and Databases
}


\section{Week 9 - Final Project Part 1
}

\section{Week 10 - Final Project Part 2
}

\begin{thebibliography}{99}

\bibitem{melissinos}
A.~C. Melissinos and J. Napolitano, \textit{Experiments in Modern Physics},
(Academic Press, New York, 2003).


\end{thebibliography}


\end{document}
